\begin{abstract}
    En este trabajo se analizan las principales limitaciones prácticas de sensores, actuadores y tareas auxiliares en un sistema de control digital basado en Arduino, aplicado al experimento \emph{ball and beam}. Se estudiaron distintos dispositivos de medición y actuación, considerando tanto su resolución como su velocidad de respuesta, y se compararon los valores teóricos con los obtenidos experimentalmente. Los resultados muestran que, si bien los sensores presentan ruido y los actuadores tienen restricciones en precisión y tiempo de respuesta, estas limitaciones se mantienen dentro de márgenes aceptables para el control en tiempo real. Se concluye que la plataforma Arduino resulta adecuada para la implementación del lazo de control, siempre que se tengan en cuenta las restricciones identificadas.
\end{abstract}

% vim: ts=4 sts=4 sw=4 et
